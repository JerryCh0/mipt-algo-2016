\documentclass{article}
\usepackage[T2A]{fontenc}
\usepackage[english,russian]{babel} 
\usepackage{cmap} 
\usepackage{soulutf8}
\usepackage[utf8]{inputenc}
\usepackage{amsmath}

\begin{document}
\subparagraph{\textit{ Задача ''Идеальное паросочетание "", Ткаченко Дмитрий, команда PQ.}}
\paragraph{Решение:} Докажем нашу задачу при помощи теоремы Форда-Фалкерсона. Обозначим мощность первой доли как $n$. Модифицируем наш граф: на каждом ребре введем пропускную способность 1 в направлении от вершины первой доли к вершине второй. Ведем фиктивные вершины s и t - от s проведем все ребра в вершины первой доли, а из каждой вершины второй доли - ребра в t. Получилась целочисленная сеть, то есть если доказать, что пропускная способность минимального разреза равна $n$, то по теореме Форда-Фалкерсона величина наибольшего потока будет равна $n$.
\paragraph{\Leftarrow} Понятно, что если в бинарной сети величина наибольшего потока суть $n$, то существует ровно столько же непересекающихся (по вершинам) путей из истока в сток, что нам и нужно (это и будет нашим паросочетанием). 
\newline
	Мощность минимального разреза не превышает $n$ из рассмотрения разреза, в котором множество $S$ содержит лишь вершину $s$.
\newline
	Теперь рассмотрим какой-то разрез $(S,T)$. Если в $S$ попали $m \le n$ вершин из левой доли и $l$ из правой. Тут возможны два случая.
\subparagraph{1)} $l < m$. По условию теоремы, данные $m$ вершин связаны с $m$ вершинами правой доли, а в силу $l < m$, они связаны минимум с $m - l$ вершинами правой доли, попавшими в $T$. В таком случае пропускная способность разреза рассчитывается из $n-m$ ребер, ведущих из истока в вершины левой доли, лежащие в $T$, и $l$ ребер, ведущих из вершин правой доли, лежащих в $S$, в сток, а так же из $m-l$ ребер из вершин первой доли, принадлежащих $S$ в вершины второй доли, принадлежащими $T$. Получается $(n - m) + l + (m - l) = n$.
\subparagraph{2)} $l \ge m$. Пропускная способность разреза получается из $n-m$ ребер, ведущих из $s$ в вершины первой доли, лежащие в $T$ и $l$ ребер, ведущих из вершин правой доли, лежащих в S в сток. Получаем $n - m + l = n + (l - m) \ge n$.
\paragraph{\Rightarrow} Очевидно, что если существует полное паросочетание, то для любого подмножества вершин левой доли, мощность вершин, смежных с ними, не меньше. 
\end{document}

